\documentclass{article}[12pt]
\renewcommand{\baselinestretch}{1.5}
\setlength{\parskip}{1em}

\usepackage[parfill]{parskip}
\usepackage[affil-it]{authblk}
\usepackage[space]{grffile}

\usepackage[a4paper]{geometry}
\geometry{verbose}
\usepackage{float}
\usepackage{graphicx}
\usepackage{setspace}
\usepackage{caption}

\usepackage[utf8]{inputenc}
\usepackage[english]{babel}

\usepackage{latexsym,textcomp,longtable,tabulary}
\usepackage{booktabs,array,multirow,braket}
\usepackage{amsfonts,amsmath,amssymb,mathbbol,calc,cancel}
\usepackage{subfigure,color,blindtext,enumitem,siunitx}
\usepackage[colorinlistoftodos]{todonotes}

\usepackage{mathtools}
\usepackage{url,hyperref,etoolbox}
\numberwithin{equation}{section}
\hypersetup{colorlinks=false,pdfborder={0 0 0}}

%+figure layout options
\restylefloat{figure}
\setlist{leftmargin=*,before=\setlength{\rightmargin}{\leftmargin}}
%-figure layout options

\providecommand\citet{\cite}
\providecommand\citep{\cite}
\providecommand\citealt{\cite}

\makeatletter
\makeatother

\begin{document}

\title{
	Inference and Synthesis of\\
	co-dimension one Bifurcations
}

\author{Gregory Szep$^1$, Neil Dalchau$^2$ and Attila Csikasz-Nagy$^1$}
\affil{$^1$King's College London, $^2$Microsoft Research Cambridge}
\date{\today}
\maketitle
\vspace{-25pt}
\section{Introduction and Motivation}

\begin{itemize}
\item qualitative equivalence more important
\item fitting to time courses not appropriate
\item gradient information makes optimisation tractable
\end{itemize}

\section{Problem Statement}

Suppose we would like to know the regions within paramter space
$\theta\in\mathbb{R}^N$
where the steady state of a set of differential equations
takes on a specified target form $t(p)$ along a known parameter
$p\in\mathbb{R}$. The target is a label $t\in\{0,1\}$ that indicates
whether the system is monostable or multi-stable at given $p$. We begin with a known form
for the differential equations for $u\in\mathbb{R}^M$
\begin{align}
	\partial_t u=F(u|\,p,\theta) 
\end{align}
and the set of steady states
$U(\theta) := \left\{ (u,p)\in\mathbb{R}^{M+1} : F(u|\,p,\theta)=0\right\}$
can be found in a local region numerically using
pseudo-arclength continuation along $u(s|\theta),p(s|\theta)$ \cite{} where
$s$ parametrises the one co-dimensional curve -- details in section
\ref{sec:continuation}.
In order to perform an optimisation 
we need mapping from the states $U(\theta)$ to an output that can be
compared to $t(p)$. Section \ref{sec:objective-function} motivates this mapping and
constructs the objective function.

\section{Pseudo-arclength Continuation}
\label{sec:continuation}

\begin{itemize}
    \item predictor-corrector algorithm
    \end{itemize}

\section{Objective Function}
\label{sec:objective-function}

First we note that although $U(\theta)$ contains both $u,p$ we
are only interested in how many unique $u$ there are for a given $p$.
This can be done by taking a histogram of equally sized bins along $p$.
Another way of getting a quantity that is proportional to count is
by convolution $*$ of $p(s|\theta)$ along $s$ with a symmetric kernel
$\phi(p\,|\alpha)$ with a bandwidth $\alpha$ indicating bin width.
This way we get a smooth and differentiable version of the histogram
operation.
\begin{align}
    \phi(p\,|\alpha)*U(\theta):=
    \int_{\mathbb{R}}\!\phi(p-p(s|\theta)\,|\alpha)\,\mathrm{d}s
\end{align}
The result is now proportional to count, but needs to
be mapped to the unit interval $[0,1]$ in such a way that high
count regions that indicate multi-stability are mapped to one, and
low counts which indicate monostability are mapped to zero. For
this a sigmoidal activation function $\sigma(\frac{x-\mu}{\beta})$ with
unknown smoothness $\beta$ and threshold $\mu$ will do. We
are now ready to write down the objective function
\begin{align}
	\mathcal{J}(\theta\,|\,\alpha,\beta,\mu):=                            
	\int_{\mathbb{R}}\left|\;                                                  
    t(p) - \sigma\left(\frac{\phi(p\,|\alpha)*U(\theta)-\mu}{\beta}\right)
	\,\right|^2\,\mathrm{d}p
\end{align}
where $\theta$ are the parameters to be optimised and $\alpha,\beta,\mu$
are hyper-parameters. Note that $U(\theta)$ is the only $\theta$ dependence
in the expression. When we apply
\begin{align}
    \partial_\theta\mathcal{J}
    &=-2\int_{\mathbb{R}}
    \sigma'\,\partial_\theta\left(
        \frac{\phi(p\,|\alpha)*U(\theta)-\mu}{\beta}
        \right)
    \,\mathrm{d}p\\
    &=-\frac{2}{\beta}\int_{\mathbb{R}}
    \sigma'\,
        \partial_\theta(\,\phi(p|\alpha)*U(\theta)\,)
    \,\mathrm{d}p\\
    &=\frac{2}{\beta}
    \int_{\mathbb{R}}\sigma'\!\! \int_{\mathbb{R}}\phi'\,
        \partial_\theta p(s|\theta)
    \,\mathrm{d}s \,\mathrm{d}p
\end{align}
remarkably everything is differentiable including the algorithm in section
\ref{sec:continuation}. This gradient information can be used in a
gradient descent optimization to minimize $\mathcal{J}(\theta)$.

\section{Normal Forms}
\label{sec:normal-forms}

\begin{itemize}
    \item saddle-node, pitchfork, transcritical
    \item optimisation landscape with changing hyperparams
    \item benchmarks against other algos
\end{itemize}


\section{Chemical Reaction Networks}
\label{sec:networks}

\begin{itemize}
    \item toggle switch
    \item cell cycle
    \item application to structure $\rightarrow$ function (Luca)
\end{itemize}

\section{Conclusions and Extensions}
\label{sec:conclusions}

\begin{itemize}
    \item hyperparam optimization
    \item hopf bifurcations
    \item pattern formation in pdes
\end{itemize}

\bibliography{refs}
\bibliographystyle{ieeetr}
\end{document}
