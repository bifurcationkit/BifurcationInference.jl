\documentclass{article}[12pt]
\renewcommand{\baselinestretch}{1.5}
\setlength{\parskip}{1em}

\usepackage[parfill]{parskip}
\usepackage[affil-it]{authblk}
\usepackage[space]{grffile}

\usepackage[a4paper]{geometry}
\geometry{verbose}
\usepackage{float}
\usepackage{graphicx}
\usepackage{setspace}
\usepackage{caption}

\usepackage[utf8]{inputenc}
\usepackage[english]{babel}

\usepackage{latexsym,textcomp,longtable,tabulary}
\usepackage{booktabs,array,multirow,braket}
\usepackage{amsfonts,amsmath,amssymb,mathbbol,calc,cancel}
\usepackage{subfigure,color,blindtext,enumitem,siunitx}
\usepackage[colorinlistoftodos]{todonotes}

\usepackage{mathtools}
\usepackage{url,hyperref,etoolbox}
\numberwithin{equation}{section}
\hypersetup{colorlinks=false,pdfborder={0 0 0}}

%+figure layout options
\restylefloat{figure}
\setlist{leftmargin=*,before=\setlength{\rightmargin}{\leftmargin}}
%-figure layout options

\providecommand\citet{\cite}
\providecommand\citep{\cite}
\providecommand\citealt{\cite}

\makeatletter
\makeatother

\begin{document}

\title{
	Inference and Synthesis of\\
	co-dimension one Bifurcations
}

\author{Gregory Szep$^1$, Neil Dalchau$^2$ and Attila Csikasz-Nagy$^1$}
\affil{$^1$King's College London, $^2$Microsoft Research Cambridge}
\date{\today}
\maketitle
\vspace{-25pt}
\section{Introduction and Motivation}

\begin{itemize}
\item qualitative equivalence more important
\item fitting to time courses not appropriate
\item gradient information makes optimisation tractable
\end{itemize}

\section{Problem Statement}

Suppose we would like to know the regions within an unknown parameter space $\theta\in\mathbb{R}^N$ where the steady state of a set of differential equations undergoes specified set of target bifurcations $\mathcal{D}:=\{p_1\dots p_K\}$ along a known parameter $p\in\mathbb{R}$. The differential equations for $u\in\mathbb{R}^M$
\begin{align}
	\partial_t u=F(u,p|\theta) 
\end{align}
and the set of steady states $\mathcal{U}(\theta) := \left\{ (u,p)\in\mathbb{R}^{M+1} : F(u,p|\theta) =0\right\}$ can be found in a local region numerically using
parameter continuation methods along a co-dimension one curve \cite{}.
Looking for parameters where  $\mathcal{U}(\theta)$ increases in curvature is an
indicator of the onset of fold bifurcations. Once fold points have been detected
one would simply seek to minimise the distance between the predicted bifurcations 
$\mathcal{P}(\theta):=\left\{\,p\in\mathbb{R}: \frac{\partial p}{\partial u}=0,(u,p)\in\mathcal{U}(\theta) \right\}$ and the targets $\mathcal{D}$.
\\\\

\section{Method}

\subsection{Pseudo-arclength Continuation}
\label{sec:continuation}

\begin{itemize}
    \item predictor-corrector algorithm
    \end{itemize}

\subsection{Optimization}

\subsubsection{Objective Function}
\label{sec:objective-function}

The objective function has two regimes: curvature-driven and target-driven. In
the curvature-driven regime the objective should seek to maximise the curvature
of $\mathcal{U}(\theta)$ anywhere within the target region containing points $\mathcal{D}$.
One curvature measure would be Kurtosis along $p$ - the fourth order standardised moment
\begin{align}
    \mathrm{Kurt}\left[\,\mathcal{U}(\theta)\,\right]:=
    \frac{1}{|\,\mathcal{U}(\theta)|}\sum_{(u,p)\in\mathcal{U}(\theta)}
    \left(\frac{p-\mu}{\sigma}\right)^4\qquad\qquad\quad\\
    \mathrm{where}\qquad
    \mu:=\!\!\!\!\!\!\sum_{(u,p)\in\mathcal{U}(\theta)}\!\!\!p/|\,\mathcal{U}(\theta)|
    \qquad
    \sigma^2:=\!\!\!\!\!\!\sum_{(u,p)\in\mathcal{U}(\theta)}\!\!\!(p-\mu)^2/|\,\mathcal{U}(\theta)|
\end{align}
The target-driven regime is only activated if there are a finite number of fold points $\mathcal{P}(\theta)$; in this case the objective function take a more familiar norm $|\,\mathcal{P}(\theta)-\mathcal{D}\,|^2$  which evaluates the distance between prediction
and target. The complete objective function is
\begin{align}
	\mathcal{J}(\theta):= \begin{cases}
	|\,\mathcal{P}(\theta)-\mathcal{D}\,|^2 & |\mathcal{P}(\theta)|>0 \\
	\mathbb{e}^{-\mathrm{Kurt}\left[\,\mathcal{U}(\theta)\,\right]} & \mathrm{otherwise}
	\end{cases}
\end{align}
In order to perform gradient descent on this objective we apply
\begin{align}
    \partial_\theta\mathcal{J}=\begin{cases}
	2(\mathcal{P}(\theta)-\mathcal{D})\,\partial_\theta\mathcal{P} & |\mathcal{P}(\theta)|>0 \\
	-\mathbb{e}^{-\mathrm{Kurt}\left[\,\mathcal{U}(\theta)\,\right]}\, \partial_\theta\mathcal{U}\,\mathrm{Kurt}'\left[\,\mathcal{U}(\theta)\,\right] & \mathrm{otherwise}
	\end{cases}
\end{align}
This objective function is differentiable --- except for at the boundary between
the two regimes --- if we can compute $\partial_\theta\mathcal{U}$ and $\partial_\theta\mathcal{P}$. This requires the algorithm in section \ref{sec:continuation} to be differentiable. This is where
\texttt{Zygote.jl}, a differentiable programming system that is able to take gradients of general program structures \cite{}, is applied.


\todo[inline]{Try adding a penalty for "conjugate stability"}
\todo[inline]{Comment that while reproducing the qualitative behaviour only can be desirable, sometimes it is also helpful to add quantitative desires over the values of the output. E.g. we don't care too much about [CFP], but actually, if it were below $C_\text{threshold}$, then we'd have a problem. This can be cast in terms of a multi-objective optimization problem, where each term separately gets a weighting factor. Personally, I wouldn't demonstrate that in this paper, but important to comment somewhere.}
\todo[inline]{Try applying the KDE function to the target output, as well as the model output.}

\subsubsection{Optimization details}

\begin{itemize}
    \item ADAM
\end{itemize}
\todo[inline]{Try removing momentum from the optimizer}

\section{Normal Forms}
\label{sec:normal-forms}

\begin{itemize}
    \item saddle-node, pitchfork, transcritical
    \item benchmarks against other algos
\end{itemize}


\section{Chemical Reaction Networks}
\label{sec:networks}

\begin{itemize}
    \item toggle switch
    \item cell cycle (Attila)
    \item application to structure $\rightarrow$ function (Luca)
\end{itemize}

\section{Conclusions and Extensions}
\label{sec:conclusions}

\begin{itemize}
    \item hyperparam optimization
    \item hopf bifurcations
    \item pattern formation in pdes (Neil)
\end{itemize}

\bibliography{refs}
\bibliographystyle{ieeetr}
\end{document}
